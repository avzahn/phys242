\documentclass[12pt]{article}
\usepackage{amsmath}
\usepackage{braket}
\usepackage{bbold}
\usepackage{mathrsfs}
\usepackage{graphicx}

\author{Alex Zahn}
\title{Phys 242 HW1}
\date{}

\begin{document}
\maketitle

\section{Some analytic results}

\subsubsection*{Recovering the free particle Schrodinger equation}

We're first asked to show

\[ i\hbar\frac{\partial K(b,a)}{\partial t_b} = -\frac{\hbar^2}{2m}\frac{\partial^2 K(b,a)}{\partial {x_b}^2}
\]

\noindent given that the form of \(K\) for a free particle is

\[ K(b,a) = \sqrt{\frac{m}{2\pi i \hbar(t_b -t_a)}}\mathrm{exp}\left(\frac{im(x_b-x_a)^2}{2\hbar(t_b-t_a)}\right)
\]

This can be done by direct computation:

\begin{align*}
i\hbar\frac{\partial K(b,a)}{\partial t_b} &= \frac{m(i\hbar(t_a-t_b)+m(x_b-x_a)^2)}{2\hbar(t_b-t_a)^3\sqrt{\frac{-2\pi m}{h(t_a-t_b)}}} \\
&=  -\frac{\hbar^2}{2m}\frac{\partial^2 K(b,a)}{\partial {x_b}^2}
\end{align*}

Next, we can use this to recover the free particle Schrodinger equation. Recall

\[ \psi(x,t) = \int\limits_{-\infty}^{\infty}\mathrm{dx'}K(x,t;x',t')\psi(x',t')
\]

Differentiating with respect to \(x\) or \(t\) commutes with the integral over \(x'\) so that

\begin{align*}
i\hbar\frac{\partial}{\partial t}\psi(x,t) &= \int\limits_{-\infty}^{\infty}\mathrm{dx'}i\hbar\frac{\partial}{\partial t}K(x,t;x',t')\psi(x',t') \\[6pt]
%
-\frac{h}{2m}\frac{\partial^2}{\partial x^2}\psi(x,t) &= \int\limits_{-\infty}^{\infty}\mathrm{dx'}\left(-\frac{h}{2m}\frac{\partial^2}{\partial x^2}K(x,t;x',t')\right)\psi(x',t')
\end{align*}

Having shown previously that

\[ -\frac{h}{2m}\frac{\partial^2}{\partial x^2}K(x,t;x',t') = i\hbar\frac{\partial}{\partial t}K(x,t;x',t')
\]

the above operations on \(\psi\) are equivalent and we obtain the free particle schrodinger equation.

\subsubsection*{Analytic form of the harmonic oscillator propagator}

Let's first consider the integral

\newcommand{\intxf}[1]{\int\limits_{-\infty}^{\infty}\mathrm{dx_#1}}
\newcommand{\expp}[1]{\exp\left(#1\right)}

\[I_1 = \intxf{1}\left(\frac{m}{2\pi i \hbar \epsilon}\right)^{\frac{2}{2}}\expp{\frac{im}{2\hbar\epsilon}\left((x_2-x_1)^2+(x_1-x_0)^2\right)-\frac{\omega}{2}(x_0^2+x_1^2+x_2^2)}
\]

We can expand the exponent to

\begin{align*}
\frac{i m x_0^2}{2 \epsilon  \hbar }-\frac{i m x_1
   x_0}{\epsilon  \hbar }+\frac{i m x_1^2}{\epsilon  \hbar
   }+\frac{i m x_2^2}{2 \epsilon  \hbar }-\frac{i m x_1
   x_2}{\epsilon  \hbar }-\frac{1}{2} x_0^2 \omega
   -\frac{x_1^2 \omega }{2}-\frac{x_3^2 \omega }{2}
\end{align*}

Collecting terms in powers of \(x_1\) this becomes

\begin{align*}
x_1^2 \left(\frac{i m}{\epsilon  \hbar }-\frac{\omega
   }{2}\right)+x_1 \left(-\frac{i m
   x_0}{\epsilon  \hbar }-\frac{i m x_2}{\epsilon  \hbar
   }\right)-\frac{1}{2} x_0^2 \omega -\frac{x_2^2 \omega
   }{2}+\frac{i m x_0^2}{2 \epsilon  \hbar }+\frac{i
   m x_2^2}{2 \epsilon  \hbar }\\[6pt]
%
\equiv -a_1x_1^2+b_1x_1+c_1
\end{align*}

Notice that we always have \(\mathrm{Re}(a_1) < 0\). \(I_1\) is therefore just a gaussian integral:

\begin{align*}
I_1 = \frac{m}{2\pi i \hbar \epsilon}\sqrt{\frac{\pi}{a}}e^{\frac{b_1^2}{4a_1}+c_1}
\end{align*}

Now consider

\begin{align*}
I_2 = \intxf{2}\sqrt{\frac{m}{2\pi i \hbar \epsilon}}\expp{\frac{im}{2\hbar\epsilon}(x_3-x_2)^2-\frac{\omega}{2}x_3^2}  I_1
\end{align*}

Clearly \(I_2\) can be put into the same form as \(I_1\), and we can know without doing any calculation at all that \(\mathrm{Re}(a_2)<0\) or else \(I_2\) (and our propagator) will diverge. Then \(I_2\) can be computed exactly as \(I_1\). We can repeat this reasoning for all \(I_N\), and presumably pick out a pattern in the form of \(I_N\) and next show inductively that \(I_N\) converges to the form given in the lecture notes.

It's also obvious that this procedure is going to painfully messy, and probably beyond my algebraic abilities.

So I'm going to abandon this problem and invoke ambiguity of the problem statement (see page 7 of lecture 4) to say that we're actually meant to solve the much more tractable problem of finding the classical action.

\subsubsection*{\(S_{cl}\) for the harmonic oscillator}

The classical system is described by

\[L = \frac{1}{2}m\dot{x}^2 - \frac{1}{2}m\omega^2 x^2
\]

which yields an Euler-Lagrange equation of motion given by

\[-m\omega^2 x_{cl} - m\ddot{x}_{cl} = 0
\]

with boundary conditions \(x_{cl}(t_i) = x_i\) and \(x_{cl}(t_f) = x_f\). The general solution is a linear combination of sinusoids of angular frequency \(\omega\). The particular solution that satisfies our boundary conditions is

\[x_{cl} = \frac{\sin\omega(t_f-t)}{\sin\omega(t_f-t_i)}x_i + \frac{\sin\omega(t-t_i)}{\sin\omega(t_f-t_i)}x_f
\]

where the sign of the \( t_f -t \) and \( t - t_i \) terms has been chosen so that the angular frequency of the sines remains \(+\omega\).

The classical action is straightforward to compute from here:

\begin{align*}
S_{cl} = \int\limits_{t_i}^{t_f}\mathrm{dt}L(x_{cl},\dot{x}_{cl}) = \frac{1}{2}m\omega\frac{(x_i+x_f)^2\cos\omega(t_f-t_i) - 2x_ix_f}{\sin\omega(t_f-t_i)}
\end{align*}

\section{Harmonic Oscillator Simulation}

We compute the discrete single timestep propagator in \texttt{propagor.py}'s \newline \texttt{harmonic\_oscillator.propagator()} method.

\texttt{propagator.py} provides two options for evolving a system once a propagator has been computed. \texttt{simulation.run()} will repeatedly apply the propagator to an initial wavefunction, storing the result after every iteration. Notice that this is analytically equivalent to taking \(\psi(x,t) = \Delta x^{N-1}K_{\Delta T}^{N-1}\) where \(t = N\Delta t\).

\texttt{simulation.run\_renormalize()} is equivalent except that it normalizes the wavefunction after every iteration.

\subsubsection*{Probability amplitude losses}

Running the simulation without per-iteration normalization causes large losses in to the wavepacket normalization and amplitude, as is clear from below.

\includegraphics[scale=.8]{wavefunction_timelapse.png}

Normalizing after each time step seems to improve the simulation fidelity considerably:

\includegraphics[scale=.8]{wavefunction_timelapse_renormalize.png}

\subsubsection*{Expectation values of observables}

For this section, I'll only plot observables for the renormalized run. 

First, we can demonstrate conservation of expectation energy by plotting the epxectation values of the kinetic, potential, and total energy over one period of the oscillation. The simulation does seem to recover conservation of energy. The high frequency artifact is likely due to lazy use of the built in \texttt{numpy.diff} in simulating the effect of the momentum operator.

\includegraphics[scale=.8]{energy_renormalize.png}

Position and momentum seem to have worked well also:

\includegraphics[scale=.8]{EX_EP_renormalize.png}

\subsection*{Simulation fidelity}

The assignment only asked us to do this qualitatively, mostly by noting visual consistency with the graphs in lecture 2. I was going to compute residuals of the wavefunction against the analytic solution, but this has not been a good week.
 
\end{document}

